% --更改类信息--
\documentclass[UTF8,hyperref]{sysuthss}

% 产生 originauth.tex 里的 \Square。
\usepackage{wasysym}
% 提供 verbatiminput 命令和 comment 环境。
\usepackage{verbatim}

\usepackage{pifont}

\usepackage{pdfpages}
% 设置页芯居中。
\geometry{centering}
% 设定行距。
\renewcommand{\baselinestretch}{1.5}

% 使引用标记成为上标。
\newcommand{\supercite}[1]{\textsuperscript{\cite{#1}}}
% 罗列环境中如果每个项目都只有一行左右,则会显得很松散,此时可采用这个命令。
\newcommand{\denseenum}{\setlength{\itemsep}{0pt}}

\hypersetup{
	colorlinks,
	citecolor=black,
	filecolor=black,
	linkcolor=black,
	urlcolor=black
}

\begin{document}
	% 各种文档信息。
	\renewcommand{\thesisname}{本\hspace{0.7em}科\hspace{0.7em}生\hspace{0.7em}毕\hspace{0.7em}业\hspace{0.7em}论\hspace{0.7em}文(设计)}
	% 题目一般不宜超过 20 个字。
	\title{基于****的****研究与实现}
	\etitle{The research of ***  based on ***}
	\author{郅波}
	\eauthor{Zhi Bo}
	\studentid{0******}
	\date{二〇一一年五月}
	\school{信息科学与技术学院}
	\major{计算机科学与技术}
	\emajor{Computer Science and Technology}
	\direction{****}
	\mentor{***(**)}
	\ementor{}
	% 关键词应有 3~5 个。
	\keywords{}
	\ekeywords{}

	%% 以下为正文之前的部分,页码为小写罗马数字,但不显示页眉和页脚。
	\frontmatter\pagenumbering{roman}\pagestyle{empty}

	% --这里初步改好了一个封面,不标准TODO--
	\maketitle
	% --可以使用includepdf直接将转换好的附表或封面加入--
	%\includepdf[pages=1-2]{chap/cover.pdf}
	%\includepdf[pages=1-2]{chap/report.pdf}
	%\includepdf[pages=1-2]{chap/check.pdf}
	%\includepdf[pages=1-2]{chap/reply.pdf}
	%\includepdf[pages=1-2]{chap/statement.pdf}

	% 版权声明。
	% \include{chap/copyright}
	% 中英文摘要。
	% 摘要要求在 300~500字 。

\cleardoublepage
\begin{cabstract}

	这个模板是从~\textit{pkuthss}~修改而来,旨在维护一个方便易用的论文模板。
\end{cabstract}

\cleardoublepage
\begin{eabstract}
	
	This template is modified from \textit{pkuthss} \ldots 
\end{eabstract}


	% 自动生成目录。
	\tableofcontents
	% --图片目录(可选)--
	% \listoffigures
	% --表格目录(可选)--
	% \listoftables

	%% 以下为正文,页码为小写罗马数字,但不显示页眉和页脚。
	\mainmatter\pagenumbering{arabic}\pagestyle{fancy}

	% \CTEXsetup[format+={\left}]{section}

	% 绪言。
	%\include{chap/introduction}
	% 各章节。
	\chapter{第一章标题}
\section{测试}
测试引用\supercite{AC1}\supercite{AC4}\supercite{SEL3}

	\include{chap/chap2}
	\include{chap/chap3}
	\include{chap/chap4}
	% 结论。
	\specialchap{结\hspace{1em}论}


	\begin{appendix}
		% 参考文献。
		\bibliographystyle{chinesebst}\bibliography{sample}
		% 此命令手动地在目录中增加相当于章级别的一行。
		\addcontentsline{toc}{chapter}{参考文献}
		% 此命令和真实的一级章命令结合,从而使 \addcontentsline
		% 在目录中产生的页码正常。
		\phantomsection

		% 各附录。
		% \include{chap/encl1}
	\end{appendix}

	%% 以下为正文之后的部分,页码为大写罗马数字。
	\backmatter\pagenumbering{Roman}

	% 致谢。
	\chapter{致\hspace{1em}谢}


	% 原创性声明和使用授权说明,不显示页码。
	%\pagestyle{empty}
	%\include{chap/originauth}
	\cleardoublepage
	%--成绩评定表--
	% \includepdf[pages=1-2]{chap/grade.pdf}
\end{document}

